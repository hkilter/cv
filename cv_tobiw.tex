% !TeX program = xelatex
\documentclass{scrartcl}

% This commad is defined in the makefile to generate the different
% language versions. If one compiles it directly this fallback is active.
\providecommand{\SelectedLanguage}{german}

% This commad is defined in the makefile to select a random color scheme.
% If one compiles it directly this fallback is active.
\providecommand{\SelectedColorSchemeNumber}{5}% = random

\RequirePackage{polyglossia}
   \setmainlanguage{\SelectedLanguage}

\usepackage{microtype}

\usepackage[
%   color-mode = gray,
   color-scheme-number = \SelectedColorSchemeNumber,
%   load-fonts = false,
]{citobiw}

\usepackage[
   language = \SelectedLanguage,
%   show-frame,
   year-format = YY,
]{cvtobiw}

\usepackage{hyperref}
   \urlstyle{same}

\begin{document}

\begin{languagecontent}{german}
   \setvar{main}{%
      \textcp{\textbf{Hallo.}}
      
      Darf ich mich Ihnen Vorstellen? Ich bin Tobias Weh, Lehrer und freier Grafiker
      aus Osnabrück. Neben der Physik und der Musik schlägt mein Herz vor allem für
      Buchstaben, Typografie und natürlich gute Gestaltung. \emph{Design} bedeutet
      für mich in erster Linie, das Lösen von einem konkreten Problem, um damit Ihnen
      oder Ihren Kunden das Leben in gewisser Weise, zum Beispiel durch eine schnell
      und einfach zu erfassende Darstellung von Informationen, zu erleichtern. Erst
      wenn so eine Lösung gefunden ist, kann man sie mit gestalterischen Mitteln und
      sauberem Handwerk in eine ästhetische Form bringen und damit eine nachhaltige
      und gute Gestaltung erreichen.
      
      Ich stehe Ihnen gerne zur Verfügung, um mit Ihnen gemeinsam ein Erscheinungsbild
      (\emph{Corporate Identity}) zu entwickeln, für Ihre Texte oder Noten ein
      angemessenes Layout zu finden und sie zu setzen, für Ihr Buch anschauliche
      Abbildungen anzufertigen oder um Sie in allen Fragen rund um \TeX/ zu beraten.
      
      Herzliche Grüße\\
      Tobias Weh
   }

   \setvar{contact}{%
      T\kern-1ptobias W\kern-0.4pteh\\
      Spindelstraße 25\\
      40980 Osnabrück
      
      054\kern-0.2pt1\,·\,40\kern-0.1pt757837\\
      0\kern-0.2pt160\,·\,5063337
      
      \href{mailto:mail@tobiw.de}{mail@tobiw\kern-1pt.de}\\
      \href{http://tobiw.de}{tobiw\kern-1pt.de}
   }

   \setvar{skills}{
      \minisec{Sprachen}
      \skill{Deutsch}{100}\\
      \skill{Englisch}{75}
      
      \minisec{Programme}
      Adobe CC (InDesign, InCopy,\linebreak Illustrator, Photoshop),\linebreak
      Microsoft Office (Word, Excel, Powerpoint),
      Pages, Keynote,
      \TeX/,
      Terminal/Konsole,
      Sibelius, Finale
      
      \minisec{Programmiersprachen}
      \skill{\LaTeXe/}{100}\\
      \skill{\LaTeXiii/}{95}\\
      \skill{HTML5}{85}\\
      \skill{CSS3}{85}\\
      \skill{PHP}{70}\\
      \skill{Java}{55}\\
      \skill{JavaScript}{40}
      
      \minisec{Systeme}
      \skill{Mac OS X}{100}\\
      \skill{Linux (Ubuntu)}{65}\\
      \skill{Windows}{45}
   }

   \setvar{timeline-entries}{
      \timelineentry{1988}{geboren in Stadthagen bei Hannover}{}[28]
      \timelineentry{2007}{Abitur am Fachgymnasium Technik, Stadthagen}
         {Leistungskurse: Deutsch und Elektrotechnik}
      \timelineentry{2007-2008}[14]{Freiwilliges Soziales Jahr Kultur}
         {Im Rahmen meines FSJ beim Landesmusikrat Niedersachsen e.\,V.
         habe ich diverse Projekte, wie den Bläserklassen-Tag oder Jugend musiziert
         mitorganisiert und\linebreak durchgeführt.}[14]
      \timelineentry{2008-2014}{Studium an der Uni Osnabrück}
         {Lehramt für Gymnasium\\(Physik, Musik)}
      \timelineentry{2011-2014}[1]{\TeX/-Kurse}
         {Im Fachbereich Physik der Uni Osnabrück habe ich \TeX/-Kurse gegeben.}
      \timelineentry{2012}[2]{Bachelorarbeit}
         {„Entwicklung des Java-Programms FIELDS zur zweidimensionalen\linebreak Darstellung
         von Feldern mithilfe des PhidgetInterfaceKit 2/2/2“}[11]
      \timelineentry{2014}[2]{Masterarbeit}
         {„Anschauliche Erklärung der trans\-mittierten und reflektierten Schallwellen
         am Ende einer offenen Röhre“}[11]
      \timelineentry{2011-today}[15]{selbstständige Tätigkeit}
         {als Grafiker, Setzer, \TeX/-Berater}
      \timelineentry{2012-2014}[16]{wissenschaftliche Hilfskraft}
         {Betreuung der Internetseite (Typo3) des Instituts für Musik der Uni Osnabrück}
      \timelineentry{2013-mid2013}[17]{wissenschaftliche Hilfskraft}
         {Tutor in der Physikdidaktik an der Uni Osnabrück}[10]
      \timelineentry{2014-today}[17]{Studium an der FH Bielefeld}
         {Bachelor Gestaltung\\(Grafik und Kommunikationsdesign)}
   }

   \setvar{code}{
      Erstellt mit \TeX/ – Code auf \href{https://github.com/tweh/cv}{github.com/tweh/cv}.
   }
\end{languagecontent}

\begin{languagecontent}{english}

   \setvar{contact}{%
      T\kern-1ptobias W\kern-0.4pteh\\
      Spindelstraße 25\\
      40980 Osnabrück\\
      Germany\\[0.5\baselineskip]
      %
      +49\,54\kern-0.2pt1\,·\,40\kern-0.1pt757837\\
      +49\,160\,·\,5063337\\[0.5\baselineskip]
      %
      \href{mailto:mail@tobiw.de}{mail@tobiw\kern-1pt.de}\\
      \href{http://tobiw.de}{tobiw\kern-1pt.de/\kern-0.5pten}
   }

   \setvar{skills}{
      \minisec{Languages}
      \skill{German}{100}\\
      \skill{English}{80}
      
      \minisec{Software}
      Adobe CC (InDesign, InCopy, Illustrator, Photoshop),
      Microsoft Office \linebreak (Word, Excel, Powerpoint),
      Pages, Keynote,
      \TeX/,
      Terminal/Cosole,\linebreak
      Sibelius, Finale
      
      \minisec{Programming Languages}
      \skill{\LaTeXe/}{100}\\
      \skill{\LaTeXiii/}{95}\\
      \skill{HTML5}{80}\\
      \skill{CSS3}{80}\\
      \skill{PHP}{75}\\
      \skill{Java}{65}\\
      \skill{JavaScript}{50}
      
      \minisec{Operating Systems}
      \skill{Mac OS X}{100}\\
      \skill{Linux (Ubuntu)}{85}\\
      \skill{Windows}{75}
   }

   \setvar{code}{
      Made with \TeX/ – Code at \href{https://github.com/tweh/cv}{github.com/tweh/cv}
   }
\end{languagecontent}

\setvar{image}{%
   \includegraphics[width=\usedim{right-col-width}]{img/tobiw}
}

\setvar{logo}{%
   \TobiWLogo[height=4ex]
}

\setvar{timeline-birth}{1988}
\setvar{timeline-start}{2007}
\setvar{timeline-end}{2015}

\MakeCV

\end{document}